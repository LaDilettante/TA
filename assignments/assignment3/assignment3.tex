\documentclass{article}\usepackage[]{graphicx}\usepackage[]{color}
%% maxwidth is the original width if it is less than linewidth
%% otherwise use linewidth (to make sure the graphics do not exceed the margin)
\makeatletter
\def\maxwidth{ %
  \ifdim\Gin@nat@width>\linewidth
    \linewidth
  \else
    \Gin@nat@width
  \fi
}
\makeatother

\definecolor{fgcolor}{rgb}{0.345, 0.345, 0.345}
\newcommand{\hlnum}[1]{\textcolor[rgb]{0.686,0.059,0.569}{#1}}%
\newcommand{\hlstr}[1]{\textcolor[rgb]{0.192,0.494,0.8}{#1}}%
\newcommand{\hlcom}[1]{\textcolor[rgb]{0.678,0.584,0.686}{\textit{#1}}}%
\newcommand{\hlopt}[1]{\textcolor[rgb]{0,0,0}{#1}}%
\newcommand{\hlstd}[1]{\textcolor[rgb]{0.345,0.345,0.345}{#1}}%
\newcommand{\hlkwa}[1]{\textcolor[rgb]{0.161,0.373,0.58}{\textbf{#1}}}%
\newcommand{\hlkwb}[1]{\textcolor[rgb]{0.69,0.353,0.396}{#1}}%
\newcommand{\hlkwc}[1]{\textcolor[rgb]{0.333,0.667,0.333}{#1}}%
\newcommand{\hlkwd}[1]{\textcolor[rgb]{0.737,0.353,0.396}{\textbf{#1}}}%

\usepackage{framed}
\makeatletter
\newenvironment{kframe}{%
 \def\at@end@of@kframe{}%
 \ifinner\ifhmode%
  \def\at@end@of@kframe{\end{minipage}}%
  \begin{minipage}{\columnwidth}%
 \fi\fi%
 \def\FrameCommand##1{\hskip\@totalleftmargin \hskip-\fboxsep
 \colorbox{shadecolor}{##1}\hskip-\fboxsep
     % There is no \\@totalrightmargin, so:
     \hskip-\linewidth \hskip-\@totalleftmargin \hskip\columnwidth}%
 \MakeFramed {\advance\hsize-\width
   \@totalleftmargin\z@ \linewidth\hsize
   \@setminipage}}%
 {\par\unskip\endMakeFramed%
 \at@end@of@kframe}
\makeatother

\definecolor{shadecolor}{rgb}{.97, .97, .97}
\definecolor{messagecolor}{rgb}{0, 0, 0}
\definecolor{warningcolor}{rgb}{1, 0, 1}
\definecolor{errorcolor}{rgb}{1, 0, 0}
\newenvironment{knitrout}{}{} % an empty environment to be redefined in TeX

\usepackage{alltt}

\usepackage{amsmath, amsthm, amsfonts}
\usepackage{enumerate}
\usepackage{hyperref}

\title{Assignment 3}
\author{Anh Le}
\IfFileExists{upquote.sty}{\usepackage{upquote}}{}
\begin{document}

\maketitle

\begin{knitrout}
\definecolor{shadecolor}{rgb}{0.969, 0.969, 0.969}\color{fgcolor}\begin{kframe}
\begin{alltt}
\hlcom{## knitr configuration: http://yihui.name/knitr/options#chunk_options}
\hlstd{opts_chunk}\hlopt{$}\hlkwd{set}\hlstd{(}\hlkwc{error}\hlstd{=} \hlnum{TRUE}\hlstd{,} \hlkwc{warning} \hlstd{=} \hlnum{FALSE}\hlstd{,} \hlkwc{message} \hlstd{=} \hlnum{FALSE}\hlstd{,}
               \hlkwc{tidy} \hlstd{=} \hlnum{FALSE}\hlstd{,} \hlkwc{cache} \hlstd{= F,} \hlkwc{echo} \hlstd{= T,}
               \hlkwc{fig.width} \hlstd{=} \hlnum{4}\hlstd{,} \hlkwc{fig.height} \hlstd{=} \hlnum{4}\hlstd{,} \hlkwc{fig.align}\hlstd{=}\hlstr{"center"}\hlstd{)}
\hlkwd{library}\hlstd{(DAAG)}
\end{alltt}


{\ttfamily\noindent\itshape\color{messagecolor}{\#\# Loading required package: lattice}}\end{kframe}
\end{knitrout}

\section{Question 1: Problem 1.2 in book}
The \verb!orings! data frame gives data on the damage that had
occurred in US space shuttle launches prior to the disastrous
Challenger launch of January 28, 1986. Only the observations in rows
1, 2, 4, 11, 13, and 18 were included in the pre-launch charts used in
deciding whether to proceed with the launch.

Create a new data frame by extracting these rows from \verb!orings!,
and plot \verb!total! incidents against \verb!temperature! for this
new data frame.  Obtain a similar plot for the full data set.

\section{Question 2: Problem 1.11 in book}

Explain the output from the final \verb!table(gender)!.

\begin{knitrout}
\definecolor{shadecolor}{rgb}{0.969, 0.969, 0.969}\color{fgcolor}\begin{kframe}
\begin{alltt}
\hlstd{gender} \hlkwb{<-} \hlkwd{factor}\hlstd{(}\hlkwd{c}\hlstd{(}\hlkwd{rep}\hlstd{(}\hlstr{"female"}\hlstd{,} \hlnum{91}\hlstd{),} \hlkwd{rep}\hlstd{(}\hlstr{"male"}\hlstd{,} \hlnum{92}\hlstd{)))}
\hlkwd{table}\hlstd{(gender)}
\end{alltt}
\begin{verbatim}
## gender
## female   male 
##     91     92
\end{verbatim}
\begin{alltt}
\hlstd{gender} \hlkwb{<-} \hlkwd{factor}\hlstd{(gender,} \hlkwc{levels}\hlstd{=}\hlkwd{c}\hlstd{(}\hlstr{"male"}\hlstd{,} \hlstr{"female"}\hlstd{))}
\end{alltt}
\end{kframe}
\end{knitrout}
\begin{knitrout}
\definecolor{shadecolor}{rgb}{0.969, 0.969, 0.969}\color{fgcolor}\begin{kframe}
\begin{alltt}
\hlkwd{table}\hlstd{(gender)}
\end{alltt}
\begin{verbatim}
## gender
##   male female 
##     92     91
\end{verbatim}
\begin{alltt}
\hlstd{gender} \hlkwb{<-} \hlkwd{factor}\hlstd{(gender,} \hlkwc{levels}\hlstd{=}\hlkwd{c}\hlstd{(}\hlstr{"Male"}\hlstd{,} \hlstr{"female"}\hlstd{))}
\hlkwd{table}\hlstd{(gender)}
\end{alltt}
\begin{verbatim}
## gender
##   Male female 
##      0     91
\end{verbatim}
\begin{alltt}
\hlkwd{rm}\hlstd{(gender)}                  \hlcom{# Remove gender}
\end{alltt}
\end{kframe}
\end{knitrout}

\section{Question 3: Endogeneity}
\begin{enumerate}
\item When do we have a problem with endogeneity?
\item Show why reverse causality leads to endogeneity.
\item Discuss one empirical paper -- what is the dependent variable, the independent variable. Is there a potential endogeneity problem? Of what kind (ommited variable bias, selection bias, reverse causality)?
\item Fun fact: Endogeneity can also be caused by measurement error and simultaneity bias.
\end{enumerate}
\end{document}
