\documentclass{article}\usepackage[]{graphicx}\usepackage[]{color}
%% maxwidth is the original width if it is less than linewidth
%% otherwise use linewidth (to make sure the graphics do not exceed the margin)
\makeatletter
\def\maxwidth{ %
  \ifdim\Gin@nat@width>\linewidth
    \linewidth
  \else
    \Gin@nat@width
  \fi
}
\makeatother

\definecolor{fgcolor}{rgb}{0.345, 0.345, 0.345}
\newcommand{\hlnum}[1]{\textcolor[rgb]{0.686,0.059,0.569}{#1}}%
\newcommand{\hlstr}[1]{\textcolor[rgb]{0.192,0.494,0.8}{#1}}%
\newcommand{\hlcom}[1]{\textcolor[rgb]{0.678,0.584,0.686}{\textit{#1}}}%
\newcommand{\hlopt}[1]{\textcolor[rgb]{0,0,0}{#1}}%
\newcommand{\hlstd}[1]{\textcolor[rgb]{0.345,0.345,0.345}{#1}}%
\newcommand{\hlkwa}[1]{\textcolor[rgb]{0.161,0.373,0.58}{\textbf{#1}}}%
\newcommand{\hlkwb}[1]{\textcolor[rgb]{0.69,0.353,0.396}{#1}}%
\newcommand{\hlkwc}[1]{\textcolor[rgb]{0.333,0.667,0.333}{#1}}%
\newcommand{\hlkwd}[1]{\textcolor[rgb]{0.737,0.353,0.396}{\textbf{#1}}}%

\usepackage{framed}
\makeatletter
\newenvironment{kframe}{%
 \def\at@end@of@kframe{}%
 \ifinner\ifhmode%
  \def\at@end@of@kframe{\end{minipage}}%
  \begin{minipage}{\columnwidth}%
 \fi\fi%
 \def\FrameCommand##1{\hskip\@totalleftmargin \hskip-\fboxsep
 \colorbox{shadecolor}{##1}\hskip-\fboxsep
     % There is no \\@totalrightmargin, so:
     \hskip-\linewidth \hskip-\@totalleftmargin \hskip\columnwidth}%
 \MakeFramed {\advance\hsize-\width
   \@totalleftmargin\z@ \linewidth\hsize
   \@setminipage}}%
 {\par\unskip\endMakeFramed%
 \at@end@of@kframe}
\makeatother

\definecolor{shadecolor}{rgb}{.97, .97, .97}
\definecolor{messagecolor}{rgb}{0, 0, 0}
\definecolor{warningcolor}{rgb}{1, 0, 1}
\definecolor{errorcolor}{rgb}{1, 0, 0}
\newenvironment{knitrout}{}{} % an empty environment to be redefined in TeX

\usepackage{alltt}

\usepackage{amsmath, amsthm, amsfonts}
\usepackage{enumerate}
\usepackage{hyperref}

\title{Assignment 2}
\author{Anh Le}
\IfFileExists{upquote.sty}{\usepackage{upquote}}{}
\begin{document}

\maketitle

\begin{knitrout}
\definecolor{shadecolor}{rgb}{0.969, 0.969, 0.969}\color{fgcolor}\begin{kframe}
\begin{alltt}
\hlcom{## knitr configuration: http://yihui.name/knitr/options#chunk_options}
\hlstd{opts_chunk}\hlopt{$}\hlkwd{set}\hlstd{(}\hlkwc{comment} \hlstd{=} \hlstr{""}\hlstd{,} \hlkwc{error}\hlstd{=} \hlnum{TRUE}\hlstd{,} \hlkwc{warning} \hlstd{=} \hlnum{FALSE}\hlstd{,} \hlkwc{message} \hlstd{=} \hlnum{FALSE}\hlstd{,}
               \hlkwc{tidy} \hlstd{=} \hlnum{FALSE}\hlstd{,} \hlkwc{cache} \hlstd{= F,} \hlkwc{echo} \hlstd{= T,}
               \hlkwc{fig.width} \hlstd{=} \hlnum{4}\hlstd{,} \hlkwc{fig.height} \hlstd{=} \hlnum{4}\hlstd{,} \hlkwc{fig.align}\hlstd{=}\hlstr{"center"}\hlstd{)}
\end{alltt}
\end{kframe}
\end{knitrout}


\section{Question 1}

Let's pursue further the in-class example of ordering / selecting one variable based on another. We have the following mock data.
\begin{knitrout}
\definecolor{shadecolor}{rgb}{0.969, 0.969, 0.969}\color{fgcolor}\begin{kframe}
\begin{alltt}
\hlstd{mock_data} \hlkwb{<-} \hlkwd{data.frame}\hlstd{(}\hlkwc{country}\hlstd{=}\hlkwd{c}\hlstd{(}\hlstr{"US"}\hlstd{,} \hlstr{"UK"}\hlstd{,} \hlstr{"South Africa"}\hlstd{,} \hlstr{"Liberia"}\hlstd{),}
                         \hlkwc{region}\hlstd{=}\hlkwd{c}\hlstd{(}\hlstr{"America"}\hlstd{,} \hlstr{"Europe"}\hlstd{,} \hlstr{"Africa"}\hlstd{,} \hlstr{"Africa"}\hlstd{),}
                         \hlkwc{gdppc}\hlstd{=}\hlkwd{c}\hlstd{(}\hlnum{40000}\hlstd{,} \hlnum{35000}\hlstd{,} \hlnum{25000}\hlstd{,} \hlnum{9000}\hlstd{),}
                         \hlkwc{stringsAsFactors}\hlstd{=}\hlnum{FALSE}\hlstd{)}
\hlstd{mock_data}
\end{alltt}
\begin{verbatim}
       country  region gdppc
1           US America 40000
2           UK  Europe 35000
3 South Africa  Africa 25000
4      Liberia  Africa  9000
\end{verbatim}
\end{kframe}
\end{knitrout}

What I showed you in class is to select the \verb`country` variable based on the \verb`gdppc` variable, like so:
\begin{knitrout}
\definecolor{shadecolor}{rgb}{0.969, 0.969, 0.969}\color{fgcolor}\begin{kframe}
\begin{alltt}
\hlcom{# Get countries with GDP per capita > 10000}
\hlstd{mock_data}\hlopt{$}\hlstd{country[mock_data}\hlopt{$}\hlstd{gdppc} \hlopt{>} \hlnum{10000}\hlstd{]}
\end{alltt}
\begin{verbatim}
[1] "US"           "UK"           "South Africa"
\end{verbatim}
\begin{alltt}
\hlcom{# Get countries with above-average GDP per capita}
\hlstd{mock_data}\hlopt{$}\hlstd{country[mock_data}\hlopt{$}\hlstd{gdppc} \hlopt{>} \hlkwd{mean}\hlstd{(mock_data}\hlopt{$}\hlstd{gdppc)]}
\end{alltt}
\begin{verbatim}
[1] "US" "UK"
\end{verbatim}
\end{kframe}
\end{knitrout}

Now, the question is how to select countries that have \verb`gdppc > 10000` AND belong in Africa? Phrased more generally, how do we subset the data frame using two / multiple conditions? (Google if you don't know how -- I have phrased the question in very Google-able terms)

So here's your first assignment.
\begin{enumerate}
  \item Using the mock data above, select Africans countries that have \verb`gdppc > 10000`.
  \item Download real data from package \verb`WDI`, the subset the data according to some conditions that interests you. (E.g. List all African countries that have below / above average GDP per capita; What about other continents? Variables other than GDP, etc.)
\end{enumerate}


\section{Question 2 -- Problem 1.9.1 in the book}

This problem involves data frame -- re-read the book chapter on data frame if necessary
\begin{knitrout}
\definecolor{shadecolor}{rgb}{0.969, 0.969, 0.969}\color{fgcolor}\begin{kframe}
\begin{alltt}
\hlkwd{library}\hlstd{(DAAG)} \hlcom{# install if you have it yet}
\end{alltt}
\end{kframe}
\end{knitrout}

The following table gives the size of the floor area (ha) and
  the price (\$000), for 15 houses sold in the Canberra (Australia)
  suburb of Aranda in 1999.


\begin{knitrout}
\definecolor{shadecolor}{rgb}{0.969, 0.969, 0.969}\color{fgcolor}\begin{kframe}
\begin{alltt}
\hlstd{houseprices}
\end{alltt}
\begin{verbatim}
   area bedrooms sale.price
9   694        4      192.0
10  905        4      215.0
11  802        4      215.0
12 1366        4      274.0
13  716        4      112.7
14  963        4      185.0
15  821        4      212.0
16  714        4      220.0
17 1018        4      276.0
18  887        4      260.0
19  790        4      221.5
20  696        5      255.0
21  771        5      260.0
22 1006        5      293.0
23 1191        6      375.0
\end{verbatim}
\end{kframe}
\end{knitrout}

\begin{enumerate}
\item Plot \verb+sale.price+ versus \verb+area+.
\item Use the \verb+hist()+ command to plot a histogram of the sale prices.
\item Repeat (a) and (b) after taking logarithms of sale prices.
\item The two histograms emphasize different parts of the range of sale prices. Describe the differences.
\end{enumerate}

\end{document}
